\documentclass{article}
\usepackage{titlesec} %for styling of sections
\usepackage{dsfont} %for \mathds{1}
\usepackage{hyperref} %for url references
\usepackage[margin=0.5in]{geometry} %for narrow margins
\usepackage{amsthm}
\newtheorem{definition}{Definition}
\newtheorem{example}{Example}[section]

\titleformat{\section}
    {\normalfont\Large\bfseries}
    {\thesection}
    {20pt}
    {\Large}
    [{\titlerule[0.8pt]}]

\titleformat{\subsection}
    {\normalfont\Large\bfseries}
    {\thesubsection}
    {20pt}
    {\Large}
    %[{\titlerule}]
    [\vskip-10pt{\makebox[\linewidth][l]{\rule{0.75\textwidth}{0.5pt}}}]

\titleformat{\subsubsection}
    {\normalfont\large\bfseries}
    {\thesubsubsection}
    {20pt}
    {\large}
    %[\vskip-10pt{\makebox[\linewidth][l]{\rule{0.5\textwidth}{0.5pt}}}]


\title{Chi-square test}
\author{Grzegorz Karas}

\begin{document}

\maketitle

%\tableofcontents

\section{Definition \cite{wiki, paper}}
The Pearson's $\chi^2$-test is a test used for hypothesis concerning 
probabilities of multinomial random variables. It tests a null hypothesis stating that the frequency distribution of certain 
events observed in a sample is consistent with a particular theoretical distribution.

The test statistic is of following kind
\begin{equation}
    \chi^2=\sum_{i=1}^n \frac{\left(O_i-E_i\right)^2}{E_i} = N \sum_{i=1}^n \frac{\left(O_i/N-p_i\right)^2}{p_i}
\end{equation}
where
\begin{itemize}
    \item $\chi^2$ = Pearson's cumulative test statistic, which asymptotically approaches a $\chi^2$ distribution.
    \item $O_i$ = number of observation of type i
    \item N = total number of observations
    \item $E_i = Np_i$ = the expected (theoretical) count of type i
    \item n = the number of cells in the table ($rows\cdot columns$)
\end{itemize}

If the test is fasle, then the appropriate test statistic has approximately 
a noncentral $\chi^2$ distribution with the same degrees of freedom $df$ and
a noncentrality parameter $\lambda$, which depends on alternative considered.

\section{Test types}
Pearson's chi-squared test is used to assess three types of comparison: 
goodness of fit, homogeneity, and independence.

\begin{itemize}
    \item A test of \textbf{goodness of fit} establishes whether an observed frequency 
    distribution differs from a theoretical distribution.

    $\chi^2$ test statistic with degrees of freedom $df=n-1$
    \item A test of \textbf{homogeneity} compares the distribution of counts for 
    two or more groups using the same categorical variable.

    $\chi^2$ test statistic with degrees of freedom $df=(rows-1)\cdot(columns-1)$
    \item A test of \textbf{independence} assesses whether observations consisting of 
    measures on two variables, expressed in a contingency table, are independent of each other.
    
    $\chi^2$ test statistic with degrees of freedom $df=(rows-1)\cdot(columns-1)$
\end{itemize}



\subsection{Test of goodness of fit}

\subsection{Test of independence}

\subsection{Test of homogenity}

\subsection{Sample size}


\begin{thebibliography}{9}
    \bibitem{wiki} 
    Chi-squared-test 
    \url{https://en.wikipedia.org/wiki/Chi-squared\_test}.
    \bibitem{paper} 
    Guenther, W. (1977). 
    \textit{Power and Sample Size for Approximate Chi-Square Tests.} 
    The American Statistician, 31(2), 83-85.

\end{thebibliography}   

\end{document}
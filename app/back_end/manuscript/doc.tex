\documentclass{article}
\usepackage{titlesec} %for styling of sections
\usepackage{dsfont} %for \mathds{1}
\usepackage{amsfonts} %for \mathbb{R}
\usepackage{amsmath} % for \begin{align}
\usepackage{hyperref} %for url references
\usepackage[margin=0.5in]{geometry} %for narrow margins
\usepackage{amsthm}
% --------------------------------------------------
% for increasing the space before and after equation
\makeatletter
\g@addto@macro\normalsize{
  \setlength\abovedisplayskip{10pt}
  \setlength\belowdisplayskip{10pt}
  \setlength\abovedisplayshortskip{10pt}
  \setlength\belowdisplayshortskip{10pt}
}
% --------------------------------------------------
\makeatother
\newtheorem{definition}{Definition}
\newtheorem{assumption}{Assumption}
\newtheorem{example}{Example}[section]

\titleformat{\section}
    {\normalfont\Large\bfseries}
    {\thesection}
    {20pt}
    {\Large}
    [{\titlerule[0.8pt]}]

\titleformat{\subsection}
    {\normalfont\Large\bfseries}
    {\thesubsection}
    {20pt}
    {\Large}
    %[{\titlerule}]
    [\vskip-10pt{\makebox[\linewidth][l]{\rule{0.75\textwidth}{0.5pt}}}]

\titleformat{\subsubsection}
    {\normalfont\large\bfseries}
    {\thesubsubsection}
    {20pt}
    {\large}
    %[\vskip-10pt{\makebox[\linewidth][l]{\rule{0.5\textwidth}{0.5pt}}}]


\title{Chi-square test}
\author{Grzegorz Karas}

\begin{document}

\maketitle

%\tableofcontents

\section{Definition \cite{wiki, paper}}
The $\chi^2$-test is a statistical method to test hypothesis, where random variable
follows multinomial distribution. It tests a null hypothesis stating that the frequency distribution of certain 
events observed in an observer sample is consistent with a particular theoretical distribution.

The most common $\chi^2$-test is a Pearson's chi-square test in which the test statistic is of following kind
\begin{equation}
    \chi^2=\sum_{i=1}^n \frac{\left(O_i-E_i\right)^2}{E_i} = N \sum_{i=1}^n \frac{\left(O_i/N-p_i\right)^2}{p_i}
\end{equation}
where
\begin{itemize}
    \item $\chi^2$ - Pearson's cumulative test statistic, which asymptotically approaches a $\chi^2$ distribution.
    \item $O_i$ - number of observation of type i
    \item N - total number of observations
    \item $E_i = Np_i$ - the expected (theoretical) count of type i
    \item n - the number of cells in the table ($rows\cdot columns$)
\end{itemize}

If the test fails, then the appropriate test statistic has approximately 
a noncentral $\chi^2$ distribution with the same degrees of freedom ($df$) and
a noncentrality parameter $\lambda$, which depends on alternative considered.

\section{Test types \cite{wiki}}
Pearson's chi-squared test is used to assess three types of hypothesis testing: 
goodness of fit, homogeneity, and independence.

\begin{itemize}
    \item A test of \textbf{goodness of fit} establishes whether an observed frequency 
    distribution differs from a theoretical distribution.
    
    The $\chi^2$ test statistic follows the $\chi^2$ distribution with $df=n-1$.
    \item A test of \textbf{homogeneity} compares the distribution of counts for 
    two or more groups using the same categorical variable.

    The $\chi^2$ test statistic follows the $\chi^2$ distribution with $df=(rows-1)\cdot(columns-1)$.
    \item A test of \textbf{independence} assesses whether observations consisting of 
    measures on two variables, expressed in a contingency table, are independent of each other.
    
    The $\chi^2$ test statistic follows the $\chi^2$ distribution with $df=(rows-1)\cdot(columns-1)$.
\end{itemize}

\subsection{Test of goodness of fit}

Let $X=(X_1, ..., X_k)$ be a multinomial ranom variable with parameters
$n, p_1, ..., p_k$. Suppose we wish to test
\begin{equation}\label{goodness_of_fit_h_0}
    H_0: \quad p_i=p_{i}^0 \qquad i =1,2,...,k
\end{equation}
against
\begin{equation}
    H_a: \mbox{not all p's are as given by } H_0
\end{equation}
where the $p_{i}'$ are given expected numbers. The value of the chi-square test-statistic is
\begin{equation}
    \chi^2_{H_0} = \sum_{i=1}^k \frac{\left(x_i - np_{i}^0\right)^2}{np_{i}^0} = \sum_{i=1}^k \frac{\left(\widehat{p_i}  - p_{i}^0\right)^2}{p_{i}^0}
\end{equation}
The chi-square test reject $H_0$ if
\begin{equation}
    \chi^2_{H_0} > \chi^2_{k-1,1-\alpha},
\end{equation}
where $\alpha$ is the significance level, and $\chi^2_{k-1,1-\alpha}$ is the 
quantile of order $1-\alpha$ of the central $\chi^2$ distribution with $k-1$ degrees of
freedom.
The p-value of the test is
\begin{equation}
    p-value = P\left(X>\chi^2_{H_0}\right)
\end{equation}
To evaluate the power of the test let's precisely define the alternative $H_a$ as follow.
\begin{equation}\label{goodness_of_fit_h_a}
    H_a: \quad p_i=p_{i}^a \qquad i =1,2,...,k
\end{equation}
Thus the power o the test is
\begin{equation}
    Power = P^{\lambda}\left(X > \chi^2_{k-1,1-\alpha}\right)
\end{equation}
where $X$ is a ranom variable that follows the noncentral $\chi^2$ distribution with the noncentrality parameter
\begin{equation}\label{goodness_of_fit_lambda}
    \lambda = n \sum_{i=1}^k \frac{\left(p_{i}^a-p_{i}^0\right)^2}{p_{i}^0}
\end{equation}




\subsection{Test of independence}
Let $X=(X_{ij}) \in \mathbb{R}^{r \times c}$ be a multinomial random variable with
parameters $n,p_{ij}$ where $i=1,2,...,r$, $j=1,2,...,c$ and 
$\sum_{i=1}^r\sum_{j=1}^cp_{ij}=1$. 
Suppopse we wish to test independence
\begin{equation}
    H_0: \quad p_{ij}=p_{i \cdot}p_{\cdot j} \qquad i =1,2,...,r \quad j=1,2,...,c
\end{equation}
against
\begin{equation}
    H_a: \mbox{not all the equatons given under $H_0$ are satisfied}
\end{equation}
where $p_{i \cdot} = \sum_{j=1}^c p_{ij}$ and $p_{\cdot j} = \sum_{i=1}^r p_{ij}$.
The value of the chi-square test-statistic is
\begin{equation}
    \chi^2_{H_0} = \sum_{i=1}^r\sum_{j=1}^c \frac{\left(x_{ij} - x_{i \cdot}x_{\cdot j}/n\right)^2}{x_{i \cdot}x_{\cdot j}/n}
\end{equation}
where $x_{i \cdot} = \sum_{j=1}^c x_{ij}$ and $x_{\cdot j} = \sum_{i=1}^r x_{ij}$.
We observe that
\begin{align}
    \chi^2_{H_0} =\frac{1}{n} \sum_{i=1}^r\sum_{j=1}^c \frac{\left(\frac{x_{ij}}{n} - \frac{x_{i \cdot}}{n}\frac{x_{\cdot j}}{n}\right)^2}{\frac{x_{i \cdot}}{n}\frac{x_{\cdot j}}{n}}=  \frac{1}{n} \sum_{i=1}^r\sum_{j=1}^c \frac{\left(\widehat{p_{ij}} - \widehat{p_{i \cdot}}\widehat{p_{\cdot j}}\right)^2}{\widehat{p_{i \cdot}}\widehat{p_{\cdot j}}}
\end{align}
The chi-square test reject $H_0$ if
\begin{equation}
    \chi^2_{H_0} > \chi^2_{(r-1)\cdot(c-1),1-\alpha},
\end{equation}
To evaluate the power of the test let's precisely define the alternative $H_a$ as follow.
\begin{equation}
    H_a: \quad p_{ij} =\underbrace{p_{i\cdot}p_{\cdot j} + \frac{c_{ij}}{\sqrt{n}}}_{p^a_{ij}}, \qquad i =1,2,...,r \quad j=1,2,...,c, \quad where \quad \sum_{i=1}^{r}\sum_{j=1}^{c}c_{ij}=0, 
\end{equation}
Thus the power of the test is
\begin{equation}
    Power = P^{\lambda}\left(X > \chi^2_{(r-1)\cdot(c-1),1-\alpha}\right)
\end{equation}
where $X$ is a random variable that follows the noncentral $\chi^2$ distribution with the noncentrality parameter
\begin{equation}
    \lambda = \sum_{i=1}^{r}\sum_{j=1}^c \frac{c_{ij}^2}{p_{i\cdot}p_{\cdot j}} - \sum_{i=1}^{r}\frac{c_{i \cdot}^2}{p_{i \cdot}} - \sum_{j=1}^{c}\frac{c_{\cdot j}^2}{p_{\cdot j}},
\end{equation}
where $c_{i \cdot} = \sum_{j=1}^c c_{ij}$ and $c_{\cdot j} = \sum_{i=1}^r c_{ij}$.

If $\Delta_{ij} = c_{ij}/\sqrt{n}$, then 
\begin{equation}\label{independence_lambda}
    \lambda = n\left[\sum_{i=1}^{r}\sum_{j=1}^c \frac{\Delta_{ij}^2}{p_{i\cdot}p_{\cdot j}} - \sum_{i=1}^{r}\frac{\Delta_{i \cdot}^2}{p_{i \cdot}} - \sum_{j=1}^{c}\frac{\Delta_{\cdot j}^2}{p_{\cdot j}}\right],
\end{equation}



\subsection{Test of homogeneity}
Let $X_i=(X_{ij}) \in \mathbb{R}^{c}$ be a multinomial random variable with
parameters $n_{i},p_{ij}$ for $i=1,2,...,r$ and $\sum_{j=1}^cp_{ij}=1$. 
Suppose we wish to test homogeneity
\begin{equation}
    H_0: \quad p_{1j}=p_{2j}=\cdots = p_{rj} = p_{\cdot j}\quad j=1,2,...,c
\end{equation}
against
\begin{equation}
    H_a: \mbox{not all the equatons given under $H_0$ are satisfied}
\end{equation}
The value of a chi-square test-statistic is
\begin{equation}
    \chi^2_{H_0} = \sum_{i=1}^r\sum_{j=1}^c \frac{\left(x_{ij} - x_{i \cdot}x_{\cdot j}/n\right)^2}{x_{i \cdot}x_{\cdot j}/n}
\end{equation}
where $x_{i \cdot} = \sum_{j=1}^c x_{ij} = n_{i}$ and $n = \sum_{i=1}^r n_{i}$.
The chi-square test reject $H_0$ if
\begin{equation}
    \chi^2_{H_0} > \chi^2_{(r-1)\cdot(c-1),1-\alpha},
\end{equation}
To evaluate the power of the test let's precisely define the alternative $H_a$ as follow.
\begin{equation}
    H_a: p_{ij} = \underbrace{p_{\cdot j} + \frac{c_{ij}}{\sqrt{n}}}_{p^a_{ij}} , \qquad j=1,2,...,c \quad where \quad \sum_{j=1}^{c}c_{ij}=0, 
\end{equation}
Thus the power of the test is
\begin{equation}
    Power = P^{\lambda}\left(X_a > \chi^2_{(r-1)\cdot(c-1),1-\alpha}\right)
\end{equation}
where $X_a$ is a random variable that follows the noncentral $\chi^2$ distribution with the noncentrality parameter
\begin{equation}
    \lambda = \sum_{j=1}^{c}\frac{1}{p_{\cdot j}}\left[ \sum_{i=1}^{r} c_{ij}^2 \frac{n_i}{n} -  \left(\sum_{i=1}^{r} c_{ij} \frac{n_i}{n}\right)^2  \right],
\end{equation}
If $\Delta_{ij} = c_{ij}/\sqrt{n}$, then 
\begin{equation}
    \lambda = n\sum_{j=1}^{c}\frac{1}{p_{\cdot j}}\left[ \sum_{i=1}^{r} \Delta_{ij}^2 \frac{n_i}{n} -  \left(\sum_{i=1}^{r} \Delta_{ij} \frac{n_i}{n}\right)^2  \right]
\end{equation}
It is worth to observe that $\frac{n_i}{n}$ is the same as $p_{i\cdot}$ and the equation holds following form
\begin{equation}\label{homogeneity_lambda}
    \lambda = n\sum_{j=1}^{c}\frac{1}{p_{\cdot j}}\left[ \sum_{i=1}^{r} \Delta_{ij}^2 p_{i\cdot} -  \left(\sum_{i=1}^{r} \Delta_{ij} p_{i\cdot}\right)^2  \right]
\end{equation}

\section{Sample size}
The sample size required for a test to reach predefined power can be calculated
under following assumption 

\begin{assumption}
The alternative hypothesis is the one given by the observed sample. 
\end{assumption} 
In other words observed estimates construct the alternative hypothesis. The 
procedure to retrieve the sample size is following:

\begin{enumerate}
    \item Calculate contingency table in terms of probabilities ($p^a_{ij}$).
    \item For goodness-of-fit set the probabilities, for other tests calculate expected values from the contingency table ($\widehat{p_{ij}}$).
    \item Calculate deltas $\Delta_{ij}$ between contingency table and values from the set or calculated expected values from the previous step.
    \item Calculate the $\alpha$-quantile $\chi^2_{df,1-\alpha}$ of a central chi-square distribution.
    \item Having defined target power ($\beta$) of a test, find the noncentrality parameter ($\lambda$) of a noncentral chi-square distribution. 
    The parameter $\lambda$ is found solving following equation
    \begin{equation}
        \beta = P^\lambda\left(X > \chi^2_{df,1-\alpha} \right).
    \end{equation}
    \item Depending on the type of test calculate the sample size $n$. Details of the calculation is shown in the next sections.
\end{enumerate}


\subsection{Sample size - test of goodness of fit}
The $p_i^0$ from Equation \ref{goodness_of_fit_h_0} is defined a priori for every $i$.
The $p_i^a$ from Equation \ref{goodness_of_fit_h_a} is defined a posteriori and is equal to the estimate $\widehat{p_{i}}$ derived from observed sample for every $i$.
In this case the Equation \ref{goodness_of_fit_lambda} is following:

\begin{equation}
    \lambda = n \sum_{i=1}^k \frac{\left(\widehat{p_{i}}-p_{i}^0\right)^2}{p_{i}^0}
\end{equation}
So
\begin{equation}
    n = \frac{\lambda}{\sum_{i=1}^k \frac{\left(\widehat{p_{i}}-p_{i}^0\right)^2}{p_{i}^0}}
\end{equation}


\subsection{Sample size - test of independence}

Having performed steps until 5 the Equation \ref{independence_lambda} is following

\begin{equation}
    \lambda = n\left[\sum_{i=1}^{r}\sum_{j=1}^c \frac{\Delta_{ij}^2}{\widehat{p_{i\cdot}}\widehat{p_{\cdot j}}} - \sum_{i=1}^{r}\frac{\Delta_{i \cdot}^2}{\widehat{p_{i \cdot}}} - \sum_{j=1}^{c}\frac{\Delta_{\cdot j}^2}{\widehat{p_{\cdot j}}}\right],
\end{equation}
So the sample sie $n$ is
\begin{equation}
    n = \frac{\lambda}{\left[\sum_{i=1}^{r}\sum_{j=1}^c \frac{\Delta_{ij}^2}{\widehat{p_{i\cdot}}\widehat{p_{\cdot j}}} - \sum_{i=1}^{r}\frac{\Delta_{i \cdot}^2}{\widehat{p_{i \cdot}}} - \sum_{j=1}^{c}\frac{\Delta_{\cdot j}^2}{\widehat{p_{\cdot j}}}\right]},
\end{equation}

\subsection{Sample size - test of homogeneity}

Having performed steps until 5 the Equation \ref{homogeneity_lambda} is following

\begin{equation}
    \lambda = n\sum_{j=1}^{c}\frac{1}{\widehat{p_{\cdot j}}}\left[ \sum_{i=1}^{r} \Delta_{ij}^2 p_{i\cdot} -  \left(\sum_{i=1}^{r} \Delta_{ij} p_{i\cdot}\right)^2  \right]
\end{equation}
So the sample sie $n$ is
\begin{equation}
    n = \frac{\lambda}{\sum_{j=1}^{c}\frac{1}{\widehat{p_{\cdot j}}}\left[ \sum_{i=1}^{r} \Delta_{ij}^2 p_{i\cdot} -  \left(\sum_{i=1}^{r} \Delta_{ij} p_{i\cdot}\right)^2  \right]}
\end{equation}

\begin{thebibliography}{9}
    \bibitem{wiki} 
    Chi-squared-test 
    \url{https://en.wikipedia.org/wiki/Pearson\%27s\_chi-squared\_test}.
    \bibitem{paper} 
    Guenther, W. (1977). 
    \textit{Power and Sample Size for Approximate Chi-Square Tests.} 
    The American Statistician, 31(2), 83-85.

\end{thebibliography}   

\end{document}